\section{Conclusion}
Our approach of shape-based optimization for LTQP over decentralized environments consists of exploiting the structure provided by
annotated data sources with RDF shape metadata. 
We propose the usage of a shape index as a hypermedia descriptor for structural information.
This shape index maps a subdomain of the web with RDF data shapes validating its content.
We propose to solve a query-shape containment problem analogous to the classic query containment problem for dynamic source selection
using an adaptative lookup policy. 
Our preliminary results are highly promising,
as our early experiments show that this approach can significantly reduce the query execution time and the number of HTTP requests.
In future work, we will provide a complete implementation of our containment algorithm,
a formalization of the approach and a more detailed analysis of the performance of the method.
Furthermore, we intend to explore the usage of shapes in the context of LTQP for query planning and link prioritization.