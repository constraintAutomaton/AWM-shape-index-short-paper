\begin{abstract}
    % Context
    Linked Data on the Web can be considered as one very large Decentralized Knowledge Graph (DKG).
    % Need
    Querying this DKG in a link traversal-based manner is difficult due to the pseudo-infinite size of the Web,
    its unstructured nature,
    and the lack of a priori information for query planning.
    For most traversal-based queries the execution of large number of HTTP request is the bottleneck
    of the approach. 
    However in practice, queries target small subsets of the Web.
    Most of those subsets are structured either implicitly or explicitly.
    Explicit structure can be described via hypermedia descriptions.
    Using those structural information query engines can improve
    their performances by reducing their search domain.
    % Task
    Our goal is to explore the opportunities for using mappings between RDF data shapes to distributed RDF subgraph
    for the purpose of improving the performance of traversal-based queries.
    % Object
    In this article, we discuss these opportunities, present preliminary results, and discuss potential future work.
    % Findings
    Our initial experiments show that with little maintenance and work from the server,
    our method can significantly reduce the number of links traversed to answer a query leading to
    to substantial reduction in query execution time compared to the state of the art.
    % Perspectives
    In future work, we are going to formalize our method, perform more extensive experiments,
    and design algorithms for query planning that take into account this shape metadata.
    
    
\end{abstract}