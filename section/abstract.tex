\begin{abstract}
    % Context
    Linked Data on the Web can be considered as one very large Decentralized Knowledge Graph (DKG).
    % Need
    Querying this DKG in a link traversal-based manner is difficult due to the pseudo-infinite size of the Web,
    its unstructured nature,
    and the lack of a priori information for query planning.
    In practice, queries target small subsets of the Web.
    Most of those subsets are structured either implicitly or explicitly.
    Explicit structure can be announced via the path structure of the
    URL or with hypermedia descriptions.
    Using those structural information query engines can improve
    their performances.
    % Task
    Our goal is to explore the opportunities for using shape-based structural metadata within decentralized environments
    to improve discovery and query planning for traversal-based queries.
    % Object
    In this article, we discuss these opportunities, present preliminary results, and discuss potential future work.
    % Findings
    Our initial experiments show that with little maintenance and work from the server,
    our method can more selectively discover data,
    leading to significant reductions in query execution time compared to the current state of the art.
    % Conclusion
    Our preliminary results demonstrate the benefits of shape-based metadata when querying over decentralized environments.
    % Perspectives
    In future work, we are going to formalize our method, perform more extensive experiments,
    and design algorithms for query planning that take into account this shape metadata.
    
    
\end{abstract}