\begin{abstract}
    % Context
    Linked Data on the Web can be considered as one very large Decentralized Knowledge Graph.
    % Need
    While centralized query processing approaches are well-understood,
    decentralization-friendly alternatives such as Link Traversal Query Processing (LTQP)
    that don't require prior indexing are insufficiently performant for real-world use cases.
    Indeed, LTQP approaches on the web are difficult due to pseudo-infinite size of the domain,
    the unstructured nature of the medium,
    and the lack of a priori information for query planning.
    For most traversal-based queries the execution of a large number of HTTP requests is the bottleneck. 
    However, in practice, queries target small subsets of the Web.
    Web subsets are always structured either implicitly or explicitly.
    Explicit structure can be described via hypermedia descriptions.
    Using those structural information query engines can improve
    their performances by reducing their search domain.
    % Task
    Our goal is to explore the opportunities of using mappings between RDF data shapes and distributed RDF subgraphs
    for the purpose of improving the performance of traversal-based queries.
    % Object
    In this article, we discuss these opportunities, present preliminary results, and discuss potential future work.
    % Findings
    Our initial experiments show that with little maintenance and work from the server,
    our method can significantly reduce the number of links traversed to answer a query leading to
    a substantial reduction in query execution time compared to the state of the art.
    % Perspectives
    In future work, we are going to formalize our method, perform more extensive experiments,
    and design algorithms for query planning that take into consideration this shape metadata.
    
    
\end{abstract}