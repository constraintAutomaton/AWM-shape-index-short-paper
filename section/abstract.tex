\begin{abstract}
    % Context
    Data on the web is naturally unindexed and decentralized.
    Centralizing web data, particularly personal data, for real-world applications raises ethical and legal concerns.
    Yet, compared to centralized query approaches,
    decentralization-friendly alternatives such as Link Traversal Query Processing (LTQP)
    are significantly less performant and understood.
    % Need
    The two main difficulties of LTQP are the lack of apriori information about data sources and the high number of HTTP requests.
    Exploring decentralized-friendly ways to document networks of data sources could lead to solutions to alleviate those difficulties.
    RDF data shapes is the state-of-the-art for data validation of linked data documents, thus it is worthwhile to investigate
    their potential for LTQP optimization.
    % Task
    In our work, we built an early version of a query and environment aware source selection algorithm for LTQP and measured its performance in a realistic environment.
    % Object
    In this article, we present our algorithm and early results thus opening opportunities for further research for shape-based optimization of link traversal queries.
    % Findings
    Our initial experiments show that with little maintenance and work from the server, our method can reduce up to 80\% the execution time and 97\% the number of links traversed during realistic queries.
    % Perspectives
    Given our early results and the descriptive power of RDF data shapes it would be worthwhile to investigate non-heuristic-based query planning
    using RDF shapes. 
\end{abstract}
