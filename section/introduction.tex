\section{Introduction}

The World Wide Web is a naturally decentralized database.
Centralizing large segments of the web provides easier query interfaces and faster query execution time.
However, data centralization can lead to practice with ethical and legal concerns.
Thus, decentralization-friendly alternatives are relevant research topics.
The query languages webSQL~\cite{Mendelzon1996} and \href{https://www.w3.org/TR/sparql11-query/}{SPARQL} propose mechanisms to capture descentralized web data with conjunctive queries.
However, webSQL relies on web indexing~\cite{Mendelzon1996}.
Indexing processes can be expensive (particularly on the scale of the web) and necessitate frequent updates, furthermore, they can be restrictive by excluding (or not considering) some sources thus hindering the natural serendipity of the web.
SPARQL solutions rely on the publication of linked data.
Linked data in their structure (the presence of IRI) provide the possibility to find more related information without indexes.
However, most query processing using linked data is performed in the context of centralized and federated setup, thus, leaving approaches requiring no indexing to the rank of experimental methods. 

Link Traversal Query Processing (LTQP)~\cite{Hartig2012} is a method to query unindexed networks of linked data documents.
While LTQP enables live exploration of environments without prior indexing, it leads to some difficulties.
One of them is the pseudo-infinite search domain derived from the size of the World Wide Web~\cite{hartig2016walking}.
Additionally, HTTP requests can be very slow and unpredictable making their execution the bottleneck of the method~\cite{hartig2016walking}.
Reachability criteria~\cite{Hartig2012} are a partial answer to this problem by defining completeness on the traversal of URIs
contained in the internal data source of the engine instead of on the acquisition of all the results or the traversal of the whole web.
Those criteria can also be used as a lookup policy for dereferencing of external data sources.
Another difficulty is the lack of a priori information about the sources rendering query planning arduous.
To alleviate this problem, the current state-of-the-art consists of using carefully crafted heuristics for joins ordering~\cite{Hartig2011}.
Those heuristics provide non-optimal fairly performant query plans.
The limitations of this approach are usually of little importance because the main bottleneck is the high number of HTTP requests.
In response to this, current research focuses on providing fast results to the user by adequately ordering the dereferencing operations of IRIs~\cite{hartig2016walking}.

More recently LTQP research has focused on using structure in the data publication for query optimization by reducing the search domain of queries.
This line of research uses \emph{structural assumptions}~\cite{Taelman2023} to guide query engines~\cite{verborgh2020guided} towards relevant data sources.
Structural assumptions act as contracts between the data provider and the query engines stipulating that in a certain subdomain of the web, some information respecting a constraint can be found.
The usage of structural assumptions has been studied in Solid \cite{Taelman2023}.
The method involves the utilization of the 
\href{https://solidproject.org/TR/protocol#resources}{solid storage} (that we refer to as Solid pod~\cite{Taelman2023}) hypermedia description~\cite{Fielding}
to locate all the resources of a pod. 
This hypermedia description is derived from the \href{https://www.w3.org/TR/ldp/}{LDP specification}
which only captures the structures of storage but not their contents.
For query-aware optimizations, the \href{https://solid.github.io/type-indexes/}{type index specification} is additionally used.
The type index formulation proposes a more declarative approach \cite{Taelman2017} by mapping RDF classes with sets of resources.
It has been shown that by making query engines exploit those assumptions it is possible to reduce the query execution time
of realistic queries to the extent where the bottleneck is not the execution of 
HTTP requests but the suboptimal heuristic-based query plan \cite{eschauzier_quweda_2023, Taelman2023}.
Yet for multiple queries, the bottleneck remains the high number of HTTP requests  \cite{eschauzier_quweda_2023}.
It is reasonable to hypothesize that a significant portion of those HTTP requests lead to the dereferencing of
documents containing data that don't contribute to the result of the query.
Hence investigating more descriptive structural assumptions is a relevant research endeavor.

In this article, we propose to use RDF data shapes as the main mechanism for a structural assumption in the form of a shape index.
The shape index is inspired by the type index.
However, it intends to be more expressive by describing the content of the data instead of only its class \cite{Taelman2017}.
RDF data shapes are mostly used in data validation \cite{Gayo2018a} hence they provide an excellent means to describe data.
Additionally, to a lesser extent, they have been used in the context of federated query optimizations \cite{kashif2021}.
We foresee opportunities for using a shape index during data source discovery, link pruning and ordering, and query planning.
This paper presents our preliminary work on data discovery and link pruning.
