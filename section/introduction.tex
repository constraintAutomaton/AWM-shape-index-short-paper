% TeX root = ../Main.tex

\section{Introduction}
 
The large-scale publication of linked data empowers more freedom in the creation and usage of web applications.
More concretely linked data can diminish data silos \cite{Verstraete2022}
and could foster potential new forms of application ownership \cite{Mechant2021}.
The \href{https://solidproject.org/TR/protocol}{Solid} and
\href{https://docs.joinmastodon.org/}{Mastodon} protocols
are popular technologies utilizing the emancipatory potential of linked data.
Linked data is mostly modeled and published using the graph data formalization \href{https://www.w3.org/TR/rdf12-concepts/}{RDF} and its serializations.
RDF terms can be expressed using IRIs, Blank Nodes, and Litterals.
The usage of IRIs provides more reusability of data and explicitizes in a machine-interpretable way the relations between
information from multiple remote or local subgraphs.
More information about an IRI term can be found if it is dereferenceable (treated as an URL) using the HTTP protocol.


The web is a decentralized wealth of information.
Index servers and the query language webSQL \cite{Mendelzon1996} propose a mecanism to capture this knowledge using conjunctive queries.
However this approach rely on prior indexing, which can be restrictive and hinder the natural serendipity of the web. 
Link Traversal Query Processing (LTQP) \cite{Hartig2012} propose to replace the index servers by the dereferenceable IRIs 
preserving the natural link between information in the web.
LTQP starts with a few URIs called seed URIs and recursively dereferences them from an internal data source following a lookup policy.
While LTQP enables live exploration of environments without prior indexing, it leads to some difficulties.
One of them is the pseudo-infinite search domain derived from the size of the World Wide Web \cite{Hartig2014}.
Additionally, HTTP requests can be very slow and unpredictable making their execution the bottleneck of the method \cite{hartig2016walking}.
Reachability criteria \cite{Hartig2012} are a partial answer to this problem by defining completeness on the traversal of URIs
contained in the internal data source instead of on the acquisition of all the results or the traversal of the whole web.
Those criteria can also be used as a lookup policy for dereferencing of external data sources.
Another difficulty is the lack of a priori information about the sources rendering query planning arduous.
To alleviate this problem, the current state of the art is to use carefully crafted heuristics for joins ordering \cite{Hartig2011}.
Those heuristics provide non-optimal fairly performant query plans.
The limitations of this approach are usually of little importance because the main bottleneck is the high number of HTTP requests.
In response to this, current research focuses on providing fast results to the user by ordering adequately the dereferencing operations of IRIs \cite{hartig2016walking}.

Current LTQP research doesn't consider the particular structure of the querying environment.
In environement like Solid structures are describe using more or less detailed hypermedia descriptions \cite{Fielding}.
We refer to the structure exploitable for query optimizations as structural assumptions.
Structural assumptions act as contracts between the data provider and 
the query engines stipulating that in a certain subdomain of the web, some information respecting a constraint can be found.
The usage of structural assumptions has been studied in Solid \cite{Taelman2023}.
The approach consist of using those assumptions for data discovery and link prunning by guiding the
query engine towards relevant data sources \cite{verborgh2020guided}.
More concretly the approach make use of the hypermedia description of 
\href{https://solidproject.org/TR/protocol#resources}{solid storages} (colloquially referred as solid pod)
derived from the \href{https://www.w3.org/TR/ldp/}{LDP specification},
to locate all the resources of a storage.
However, the LDP specification only capture the structure of storages but not their contents.
In response, the \href{https://solid.github.io/type-indexes/}{type index specification} is additionally used.
The type index propose a more declarative approach \cite{Taelman2017} by mapping RDF classes with set of resources to perform dynamic optmizations.
It has been shown that by making query engines exploit those assumptions it is possible to reduce the query execution time
of realistic queries to the extent where the bottleneck is not the execution of 
HTTP requests but the suboptimal heuristic-based query plan \cite{eschauzier_quweda_2023, Taelman2023}.
Yet for multiple queries, the bottleneck remains the high number of HTTP requests  \cite{eschauzier_quweda_2023}.
It is reasonable to hypothesize that a significant portion of those HTTP requests lead to the dereferencing of
documents containing data that don't contribute to the result of the query.
Hence investigating more descriptive structural assumptions is a relevant research endeavor.

In this article, we propose to use RDF data shapes as the main mechanism for a structural assumption in the form of a Shape Index (SI).
The shape index is inspired by the type index.
However, it intend to be more expressive by describing the content of the data instead of only its class \cite{Taelman2017}.
% We could have used here as a reference the paper from Ben Demeester RDF Graph Validation Using Rule-Based Reasoning 
RDF data shapes are mostly used in data validation \cite{Gayo2018a} hence they provide an excellent means to describe data.
Yet to a lesser extent they have been used in the context of federated query optimization \cite{kashif2021}.
%RDF data shapes don't require significative power to be maintained because they only need to be edited 
%when the data model is modified.
%The low cost of maintenance combined with the mostly passive contribution of 
%the server when using RDF data shape for query optimization makes it a promising potential medium. 
We foresee opportunities for using a shape index during data source discovery, link pruning and ordering, and query planning.
This short paper present our preliminary work on data discovery and link pruning.