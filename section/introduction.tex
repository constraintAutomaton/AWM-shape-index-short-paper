\section{Introduction}

\sepfootnotecontent{fn:typeIndex}{
  \href{https://solid.github.io/type-indexes/}{https://solid.github.io/type-indexes/}
}

The World Wide Web is a naturally decentralized database.
Centralizing large web segments in single endpoints provides easier query interfaces and faster query execution times.
However, data centralization can lead to practices that raise ethical and legal concerns, making the exploration of decentralization-friendly query paradigms a relevant research topic.
The query languages webSQL~\cite{Mendelzon1996} and \href{https://www.w3.org/TR/sparql11-query/}{SPARQL} propose mechanisms to capture decentralized web data with conjunctive queries.
However, webSQL relies on web indexing~\cite{Mendelzon1996}.
Indexing processes can be expensive, particularly on the scale of the web, and necessitate frequent updates, furthermore, they can be restrictive by excluding some sources thus hindering the natural serendipity of the web.
SPARQL solutions rely on the publication of linked data.
Linked data in their structure particularly with the presence of IRI gives the opportunity to find more related information without indexes.
However, most query processing over linked data is performed in centralized and federated setups, leaving indexing-independent approaches largely experimental.

Link Traversal Query Processing (LTQP)~\cite{Hartig2012} is a method to query unindexed networks of linked data documents.
The method consists of starting from an initial set of IRIs concurrently answer the query provided by the user while dereferencing IRIs leading to external data sources from the internal engine triple store.
While LTQP enables live exploration of environments without prior indexing, it leads to some difficulties.
One of them is the pseudo-infinite search domain derived from the size of the World Wide Web~\cite{hartig2016walking}.
Additionally, HTTP requests can be very slow and unpredictable making their execution the bottleneck of the method~\cite{hartig2016walking}.
Reachability criteria~\cite{Hartig2012} are a partial answer to this problem by defining completeness based on the traversal of URIs
contained in the internal data source of the engine instead of on the acquisition of all the results or the traversal of the whole web.
Another difficulty is the lack of a priori information about the sources rendering query planning challenging.
To alleviate this problem, the current state-of-the-art consists of using carefully crafted heuristics for joins ordering~\cite{Hartig2011}.
The limitations of the heuristics approach are usually of little importance because the main bottleneck is the high number of HTTP requests.

Earlier LTQP research has focused on the open web.
More recently, LTQP research has shifted its focus to environments where the structure of data publication provides useful information for query optimization.
This line of research uses \emph{structural assumptions}~\cite{Taelman2023} to guide query engines~\cite{verborgh2020guided} towards relevant data sources.
Structural assumptions act as contracts between the data provider and the query engines stipulating that within a certain subdomain of the web, information meeting a specific constraint can be found.
The use of structural assumptions has been studied in Solid \cite{Taelman2023}.
The method involves the utilization of the 
\href{https://solidproject.org/TR/protocol#resources}{solid storage} hypermedia description~\cite{Fielding} to locate all the resources of a pod. 
This hypermedia description is not expressive enough to capture the content of the resources of a pod, thus, for query-aware optimizations, the \href{https://solid.github.io/type-indexes/}{type index specification}~\sepfootnote{fn:typeIndex} is additionally used.
The type index formulation proposes a more declarative approach~\cite{Taelman2017} by mapping RDF classes with sets of resources.
By using those structural assumptions it is possible to reduce the query execution time of realistic queries to the extent where the bottleneck is not the execution of HTTP requests but the suboptimal heuristic-based query plan~\cite{eschauzier_quweda_2023, Taelman2023}.
Yet, for multiple queries the high number of HTTP requests remains the main bottleneck~\cite{eschauzier_quweda_2023}.
It is reasonable to hypothesize that a significant portion of those HTTP requests lead to the dereferencing of documents containing data that do not contribute to the result of the query.
Hence, investigating more descriptive structural assumptions is a relevant research endeavor.

In this article, we propose to use RDF data shapes as the main mechanism for a structural assumption in the form of a shape index.
RDF data shapes are mostly used in data validation~\cite{Gayo2018a} thus, they provided a good formalization to describe the structure of data.
Additionally, to a lesser extent, they have been used for query optimizations~\cite{kashif2021}.
The shape index is an early effort for data summarisation of decentralized datasets~\cite{Stuckenschmidt2004,Goldman1997, Harth2010} within networks of unindexed linked data documents.
The current focus of the index is source selection.
However, we foresee opportunities to use a similar approach for link queue ordering and query planning.
This paper presents our preliminary work on data discovery and link pruning thus tackling the problem of the large search space of LTQP queries in linked data environments with structure.
